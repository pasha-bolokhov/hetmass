%\documentclass{article}
\documentclass[12pt]{article}
\usepackage{latexsym}
\usepackage{amsmath}
\usepackage{amssymb}
\usepackage{relsize}
\usepackage{geometry}
\geometry{letterpaper}

\usepackage{showlabels}

\textwidth = 6.0 in
\textheight = 8.5 in
\oddsidemargin = 0.0 in
\evensidemargin = 0.0 in
\topmargin = 0.2 in
\headheight = 0.0 in
\headsep = 0.0 in
%\parskip = 0.05in
\parindent = 0.35in


%% common definitions
\def\stackunder#1#2{\mathrel{\mathop{#2}\limits_{#1}}}
\def\beqn{\begin{eqnarray}}
\def\eeqn{\end{eqnarray}}
\def\nn{\nonumber}
\def\baselinestretch{1.1}
\def\beq{\begin{equation}}
\def\eeq{\end{equation}}
\def\ba{\beq\new\begin{array}{c}}
\def\ea{\end{array}\eeq}
\def\be{\ba}
\def\ee{\ea}
\def\stackreb#1#2{\mathrel{\mathop{#2}\limits_{#1}}}
\def\Tr{{\rm Tr}}
\newcommand{\gsim}{\lower.7ex\hbox{$
\;\stackrel{\textstyle>}{\sim}\;$}}
\newcommand{\lsim}{\lower.7ex\hbox{$
\;\stackrel{\textstyle<}{\sim}\;$}}
\newcommand{\nfour}{${\mathcal N}=4$ }
\newcommand{\ntwo}{${\mathcal N}=2$ }
\newcommand{\ntwon}{${\mathcal N}=2$}
\newcommand{\ntwot}{${\mathcal N}= \left(2,2\right) $ }
\newcommand{\ntwoo}{${\mathcal N}= \left(0,2\right) $ }
\newcommand{\none}{${\mathcal N}=1$ }
\newcommand{\nonen}{${\mathcal N}=1$}
\newcommand{\vp}{\varphi}
\newcommand{\pt}{\partial}
\newcommand{\ve}{\varepsilon}
\newcommand{\gs}{g^{2}}
\newcommand{\qt}{\tilde q}
\renewcommand{\theequation}{\thesection.\arabic{equation}}

%%
\newcommand{\p}{\partial}
\newcommand{\wt}{\widetilde}
\newcommand{\ov}{\overline}
\newcommand{\mc}[1]{\mathcal{#1}}
\newcommand{\md}{\mathcal{D}}

\newcommand{\GeV}{{\rm GeV}}
\newcommand{\eV}{{\rm eV}}
\newcommand{\Heff}{{\mathcal{H}_{\rm eff}}}
\newcommand{\Leff}{{\mathcal{L}_{\rm eff}}}
\newcommand{\el}{{\rm EM}}
\newcommand{\uflavor}{\mathbf{1}_{\rm flavor}}
\newcommand{\lgr}{\left\lgroup}
\newcommand{\rgr}{\right\rgroup}

\newcommand{\Mpl}{M_{\rm Pl}}
\newcommand{\suc}{{{\rm SU}_{\rm C}(3)}}
\newcommand{\sul}{{{\rm SU}_{\rm L}(2)}}
\newcommand{\sutw}{{\rm SU}(2)}
\newcommand{\suth}{{\rm SU}(3)}
\newcommand{\ue}{{\rm U}(1)}
%%%%%%%%%%%%%%%%%%%%%%%%%%%%%%%%%%%%%%%
%  Slash character...
\def\slashed#1{\setbox0=\hbox{$#1$}             % set a box for #1
   \dimen0=\wd0                                 % and get its size
   \setbox1=\hbox{/} \dimen1=\wd1               % get size of /
   \ifdim\dimen0>\dimen1                        % #1 is bigger
      \rlap{\hbox to \dimen0{\hfil/\hfil}}      % so center / in box
      #1                                        % and print #1
   \else                                        % / is bigger
      \rlap{\hbox to \dimen1{\hfil$#1$\hfil}}   % so center #1
      /                                         % and print /
   \fi}                                        %

%%EXAMPLE:  $\slashed{E}$ or $\slashed{E}_{t}$

%%

\newcommand{\LN}{\Lambda_\text{SU($N$)}}
\newcommand{\sunu}{{\rm SU($N$) $\times$ U(1) }}
\newcommand{\sunun}{{\rm SU($N$) $\times$ U(1)}}
\def\cfl {$\text{SU($N$)}_{\rm C+F}$ }
\def\cfln {$\text{SU($N$)}_{\rm C+F}$}
\newcommand{\mUp}{m_{\rm U(1)}^{+}}
\newcommand{\mUm}{m_{\rm U(1)}^{-}}
\newcommand{\mNp}{m_\text{SU($N$)}^{+}}
\newcommand{\mNm}{m_\text{SU($N$)}^{-}}
\newcommand{\AU}{\mc{A}^{\rm U(1)}}
\newcommand{\AN}{\mc{A}^\text{SU($N$)}}
\newcommand{\aU}{a^{\rm U(1)}}
\newcommand{\aN}{a^\text{SU($N$)}}
\newcommand{\baU}{\ov{a}{}^{\rm U(1)}}
\newcommand{\baN}{\ov{a}{}^\text{SU($N$)}}
\newcommand{\lU}{\lambda^{\rm U(1)}}
\newcommand{\lN}{\lambda^\text{SU($N$)}}
%\newcommand{\Tr}{{\rm Tr\,}}
\newcommand{\bxir}{\ov{\xi}{}_R}
\newcommand{\bxil}{\ov{\xi}{}_L}
\newcommand{\xir}{\xi_R}
\newcommand{\xil}{\xi_L}
\newcommand{\bzl}{\ov{\zeta}{}_L}
\newcommand{\bzr}{\ov{\zeta}{}_R}
\newcommand{\zr}{\zeta_R}
\newcommand{\zl}{\zeta_L}
\newcommand{\nbar}{\ov{n}}

\newcommand{\CPC}{CP($N-1$)$\times$C }
\newcommand{\CPCn}{CP($N-1$)$\times$C}

\newcommand{\lar}{\lambda_R}
\newcommand{\lal}{\lambda_L}
\newcommand{\larl}{\lambda_{R,L}}
\newcommand{\lalr}{\lambda_{L,R}}
\newcommand{\blar}{\ov{\lambda}{}_R}
\newcommand{\blal}{\ov{\lambda}{}_L}
\newcommand{\blarl}{\ov{\lambda}{}_{R,L}}
\newcommand{\blalr}{\ov{\lambda}{}_{L,R}}

\newcommand{\bgamma}{\ov{\gamma}}
\newcommand{\bpsi}{\ov{\psi}{}}
\newcommand{\bphi}{\ov{\phi}{}}
\newcommand{\bxi}{\ov{\xi}{}}

\newcommand{\ff}{\mc{F}}
\newcommand{\bff}{\ov{\mc{F}}}

\newcommand{\eer}{\epsilon_R}
\newcommand{\eel}{\epsilon_L}
\newcommand{\eerl}{\epsilon_{R,L}}
\newcommand{\eelr}{\epsilon_{L,R}}
\newcommand{\beer}{\ov{\epsilon}{}_R}
\newcommand{\beel}{\ov{\epsilon}{}_L}
\newcommand{\beerl}{\ov{\epsilon}{}_{R,L}}
\newcommand{\beelr}{\ov{\epsilon}{}_{L,R}}

\newcommand{\bi}{{\bar \imath}}
\newcommand{\bj}{{\bar \jmath}}
\newcommand{\bk}{{\bar k}}
\newcommand{\bl}{{\bar l}}
\newcommand{\bm}{{\bar m}}

\begin{document}

\begin{titlepage}

\begin{flushright}
FTPI-MINN-09/?? UMN-TH-????/09\\
%ITEP-TH-XX/09\\
June 30, 2009
\end{flushright}

\begin{center}

{\Large \bf   Heterotic \ntwoo \boldmath{${\rm CP}(N-1)$}
model \\ with twisted masses
%\\[2mm]
 }
\end{center}

\begin{center}
{\bf P. A. Bolokhov$^{a,b}$, M.~Shifman$^{c}$ and \bf A.~Yung$^{c,d}$}
\end {center}
\vspace{0.3cm}
\begin{center}

$^a${\it Physics and Astronomy Department, University of Pittsburgh, Pittsburgh, Pennsylvania, 15260, USA}\\
$^b${\it Theoretical Physics Department, St.Petersburg State University, Ulyanovskaya~1, 
	 Peterhof, St.Petersburg, 198504, Russia}\\
$^c${\it  William I. Fine Theoretical Physics Institute,
University of Minnesota,
Minneapolis, MN 55455, USA}\\
$^{d}${\it Petersburg Nuclear Physics Institute, Gatchina, St. Petersburg
188300, Russia}\\

\end{center}

\begin{abstract}

We continue the  study of heterotic non-Abelian  flux tubes (strings). Previously it was shown that the internal dynamics
of  orientational zero modes
of non-Abelian string in \ntwo QCD with U($N$) gauge group and 
$N_f=N$ fundamental matter multiplets is described by the
effective low energy two dimensional \ntwot CP$(N-1)$ model
with twisted masses on the string world sheet. On the other hand, it was shown that \none deformations of massless 
U$(N)$ bulk theory leads to  heterotic deformations
of two dimensional CP$(N-1)$ model on the string 
classically preserving
\ntwoo supersymmetry. This supersymmetry was shown to be broken 
by quantum effects. Here we combine these two results and
present a two dimensional heterotic \ntwot CP$(N-1)$ model
with twisted masses. It is supposed to describe the internal dynamics of non-Abelian strings in massive \ntwo QCD with
\none preserving deformations. We present gauge and geometric
formulations of the word sheet theory and check \ntwoo 
supersymmetry. We show that this supersymmetry is spontaneously
broken for generic masses already on the classical level.


 

\end{abstract}

\end{titlepage}

%%%%%%%%%%%%%%%%%%%%%%%%%%%%%%%%%%%%%%%%%%%%%%%%%%%%%%%%%%%%%%%%%%%%%%%%%%%%%%%%%%
%
%	  		        S E C T I O N
%
%%%%%%%%%%%%%%%%%%%%%%%%%%%%%%%%%%%%%%%%%%%%%%%%%%%%%%%%%%%%%%%%%%%%%%%%%%%%%%%%%%
\section{Introduction}

%%%%%%%%%%%%%%%%%%%%%%%%%%%%%%%%%%%%%%%%%%%%%%%%%%%%%%%%%%%%%%%%%%%%%%%%%%%%%%%%%%
%
%	  		        S E C T I O N
%
%%%%%%%%%%%%%%%%%%%%%%%%%%%%%%%%%%%%%%%%%%%%%%%%%%%%%%%%%%%%%%%%%%%%%%%%%%%%%%%%%%
\section{Gauge Formulation}
\label{gaugef}

In this section we first review two dimensional \ntwot
CP$(N-1)$ sigma model with twisted masses in the 
gauge formulation and then present its \ntwoo deformation.

\subsection{ \ntwot CP$(N-1)$  model}

As we already mentioned two dimensional
supersymmetric \ntwot CP$(N-1)$  model was shown to describe
internal dynamics of  non-Abelian strings in \ntwo
supersymmetric QCD with U$(N)$ gauge group and $N$ flavors
of quarks \cite{HT1,ABEKY,SYmon,HT2}, see
also reviews \cite{Trev,SYrev,Jrev,Trev2}. In the gauge formulation this model 
with twisted
masses $ m^l $, $ l = 1, ...\, N $ is given by the strong coupling limit $ e ~\to~ \infty $ of
the following U(1) gauge theory \cite{W79}
\begin{align}
\label{sigma22}
% 
\notag
 	\mc{L}_\text{1+1} & ~~=~~
	\,\frac{1}{4e^2}\,F_{kl}^2  ~+~ \frac{1}{e^2} \left|\p_k \sigma\right|^2 
	~+~ \frac{1}{2e^2}\, D^2
	~+~ \frac{1}{e^2}\, \blar\, i\p_L\, \lar  ~+~  \frac{1}{e^2} \blal\, i\p_R\, \lal
	\\[2mm]
%
\notag
	&
	\hspace{-0.4cm}
	~+~ 2\beta \biggl \lgroup
	\left| \nabla n \right|^2  ~+~ 2 \Bigl| \sigma - \frac{m^l}
	{\sqrt{2}} \Bigr|^2 \left| n^l \right|^2
	~+~ iD \left( \left|n^l \right|^2 - 1 \right)
	\\
%
\notag
	&
	\hspace{-0.4cm}\phantom{2\beta\lgroup}
	~+~ \bxir\, i\nabla_L \xir  ~+~ \bxil\, i\nabla_R \xil ~+~
	i\sqrt{2}\, \Bigl( \sigma - \frac{m^l}{\sqrt{2}} \Bigr) \ov{\xi}{}_{Rl} \xi_L^l
	~+~ i\sqrt{2}\, \Bigl( \ov{\sigma} - \frac{\ov{m}{}^l}{\sqrt{2}} \Bigr) \ov{\xi}{}_{Ll} \xi_R^l
	\\[2mm]
%
	&
	\hspace{-0.4cm}\phantom{2\beta\lgroup}
	~+~ i\sqrt{2}\, \ov{\xi_{[R}\, \lambda}{}_{L]}\, n
	~-~ i\sqrt{2}\, \nbar\,  \lambda_{[R}\, \xi_{L]}
 ,
	\\[1mm]
%
\notag
	&
	l~=~1,...\,N,
\end{align}
where
\beq
\nabla_k=\p_k-iA_k, \qquad \nabla_{R,L}=\nabla_0\pm i\nabla_3,
\qquad \lambda_{[R}\, \xi_{L]}= \lambda_R\xi_L-\lambda_L\xi_R
\eeq
while $x_k$, $k=0,3$ denote world sheet coordinates
on the string (we assume that the string is stretched in the
$x_3$ direction). Here $n^l$ are $N$ CP$(N-1)$ model 
complex scalar fields
and $\xi^l_{R,L}$ are their fermionic superpartners.
All fields of the gauge supermultiplet, namely
the gauge field $A_k$, complex scalar $\sigma$, fermions
$\lambda_{R,L}$ and axillary field $D$ are not dynamical
in the strong coupling limit $e^2\to\infty$. They can be 
excluded via algebraic equations of motion. In particular,
integration over $D$ and $\lambda$ give CP$(N-1)$ model
constraints
\beq
|n^l|^2=1, \qquad \bar{n}_l\xi^l_{R,L}=0.
\eeq

	We need to comment on our summation conventions for the CP($N-1$) indices $l$, {\it etc}, since
	it becomes non-trivial once the masses are introduced.
	The sum in $l$ is always implied if the index is written more than once.
	In the places where this can cause ambiguity we put the summation sign explicitly.
	Furthermore, we specify the range of the variation of CP($N-1$) indices
	in the end of equations.
	Finally, we omit the summation sign in those terms where the sum is obvious,
	such as the kinetic terms $ \ov{\xi}\, i\nabla \xi $.
	
	To prove supersymmetry of \eqref{sigma22} it is easier to include the factor of $ 2\beta $ into 
	the bracket by redefining the corresponding fields, while to determine the correspondence between the 
	above model and the geometric formulation of CP$(N-1)$ model with twisted masses it is easier to leave
	it outside, so that $ |n^l|^2 = 1 $.
	
     The model (\ref{sigma22}) ( the
     orientational sector of the world sheet theory
     on the string),
      apart from the 2-dimensional FI term $ - i D $ is nothing but
        the dimensionally reduced \none four-dimensional SQED.
        From this fact one obtains the transformation properties of the component fields under 
        \ntwot supersymmetry:
\begin{align}
%
\notag
  \delta A_{R,L} & ~~=~~ ~~~~~ 2i \lgr  \eerl\, \blarl
                              ~-~ \beerl\, \larl \rgr  , \\
%
\notag
  \delta\sigma & ~~=~~ -\sqrt{2}
                              \lgr \eer\, \blal ~-~ \beel\,\lar \rgr , 
                              \\
%
\notag
  \delta\ov{\sigma} & ~~=~~ +\sqrt{2}
                              \lgr \beer\,\lal ~-~ \eel\,\blar \rgr ,
                              \\
%
\notag
  \delta\lar & ~~=~~ -\,\eer\cdot D ~-~ \frac{1}{2}\,\eer\cdot F_{RL} 
                     ~-~ i\sqrt{2}\, \p_R\sigma \cdot \eel
                     \\
%
\notag
  \delta\lal & ~~=~~ -\,\eel\cdot D ~+~ \frac{1}{2}\,\eel\cdot F_{RL} 
                     ~-~ i\sqrt{2}\, \p_L\sigma \cdot \eer
                     \\
%
\notag
  \delta\blar & ~~=~~ -\,\beer\cdot D ~+~ \frac{1}{2}\,\beer\cdot F_{RL} 
                     ~+~ i\sqrt{2}\, \p_R\ov{\sigma} \cdot \beel
                     \\
%
\notag
  \delta\blal & ~~=~~ -\,\beel\cdot D ~-~ \frac{1}{2}\,\beel\cdot F_{RL}
                     ~+~ i\sqrt{2}\, \p_L\sigma \cdot \beer
                     \\
%
\notag
  \delta D & ~~=~~ ~~~i\,\eer\, \p_L \blar ~+~ i\,\beer\, \p_L\lar 
                   ~+~ i\,\eel\, \p_R \blal ~+~ i\,\beel\,\p_R\,\lal
                   \\
%
\label{msusy}
  \delta n & ~~=~~ -\,\sqrt{2}\, \epsilon_{[R}\, \xi_{L]} 
                   \\
%
\notag
  \delta\ov{n} & ~~=~~ +\,\sqrt{2}\,\ov{\epsilon_{[R}\, \xi}{}_{L]}
                   \\
%
\notag
  \delta\xi_R^l & ~~=~~
     -\, i\sqrt{2}\, \beel\, \nabla_R\, n^l ~+~ \sqrt{2}\, \eer\, F^l 
     ~-~ 2i\, \beer\, \Bigl(\sigma - \frac{m^l}{\sqrt{2}}\Bigr)\, n^l
     \\
%
\notag
  \delta\xi_L^l & ~~=~~
     +\, i\sqrt{2}\, \beer\, \nabla_L\, n^l ~+~ \sqrt{2}\, \eel\, F^l
     ~+~ 2i\, \beel\, \Bigl(\ov{\sigma} - \frac{\ov{m}{}^l}{\sqrt{2}}\Bigr)\, n^l
     \\
%
\notag
  \delta\ov{\xi}{}_{lR} & ~~=~~
     +\, i\sqrt{2}\, \eel\, \nabla_R\, \ov{n}{}_l 
     ~+~ \sqrt{2}\, \beer\, \ov{F}{}_l 
     ~-~ 2i\, \eer\, \Bigl(\ov{\sigma} - \frac{\ov{m}{}^l}{\sqrt{2}}\Bigr)\, \ov{n}{}_l
     \\
%
\notag
  \delta\ov{\xi}{}_{lL} & ~~=~~
     -\, i\sqrt{2}\, \eer\, \nabla_L\, \ov{n}{}_l
     ~+~ \sqrt{2}\, \beel\, \ov{F}{}_l
     ~+~ 2i\, \eel\, \Bigl(\sigma - \frac{m^l}{\sqrt{2}}\Bigr)\, \ov{n}{}_l
     \\
%
\notag
  \delta F^l & ~~=~~
     -\,i\sqrt{2} \lgr \beel\, \nabla_R\, \xi_L^l ~+~ \beer\, \nabla_L \xi_R^l \rgr
     ~-~ 2i\, \ov{\epsilon_{[R}\, \lambda}{}_{L]}\, n^l
     \\
\notag
     & \phantom{~~=~~}
     ~-~ 2i \lgr \beer\, \Bigl(\sigma - \frac{m^l}{\sqrt{2}}\Bigr)\, \xi_L^l 
             ~+~ \beel\, \Bigl(\ov{\sigma} - \frac{\ov{m}{}^l}{\sqrt{2}}\Bigr)\, \xi_R^l \rgr 
     \\
%
\notag
  \delta \ov{F}{}_l & ~~=~~
     -\,i\sqrt{2} \lgr \eer\, \nabla_L\, \ov{\xi}{}_{lR} ~+~ 
                       \eel\, \nabla_R\, \ov{\xi}{}_{lL} \rgr
     ~+~ 2i\,\ov{n}{}_l\, \epsilon_{[R}\, \lambda_{L]} 
     \\
\notag
     & \phantom{~~=~~}
     ~+~ 2i \lgr \eel \Bigl(\sigma - \frac{m^l}{\sqrt{2}}\Bigr)\, \ov{\xi}{}_{lR} 
             ~+~ \eer \Bigl(\ov{\sigma} - \frac{\ov{m}{}^l}{\sqrt{2}}\Bigr)\, 
                            \ov{\xi}_{lL} \rgr ,
\end{align}
where $F^l$ are $F$-components of $(n^l,\xi^l)$
supermultiplet, while $F_{RL}=-2i\,F_{03}$ is a
convenient notation for the gauge field
strength.

	Obviously, with vanishing masses the theory \eqref{sigma22} is invariant
	under the massless version of transformations \eqref{msusy}.
	The masses themselves can be considered just as constant gauge fields of
	the ``original'' four-dimensional theory, directed in the $ (x_1,\, x_2) $-plane \cite{HaHo,Dorey}.
	After the dimensional reduction they become constant ``$\sigma$'s''.
	Hence supersymmetry should be preserved by twisted mass deformations\footnote{
          A vector supermultiplet $(A_k,\,\sigma,\,\lambda,\,D)$ with only 
          $ \sigma $ nonvanishing and yet constant, is obviously invariant under \ntwot supersymmetry.}, 
	and it indeed is.

 

\subsection{ Heterotic \ntwoo \CPC  model}

For the \ntwo supersymmetric bulk theory the translational
sector of the effective theory on the non-Abelian string 
decouples from the orientational one. The translational sector
associated with position of the string in the $(x_1,x_2)$
plane $x_{0i}$, $i=1,2$ and its fermionic superpartners
$\zeta_L$ and $\zeta_R$. Orientational sector is described by the \ntwot CP$(N-1)$ model (\ref{sigma22}). Once \ntwo
breaking deformation is added in the bulk theory this is no longer true \cite{Edalati}. The translational sector becomes
mixed with the orientational one. In fact, fields $x_{0i}$ and 
$\zeta_L$ are still free and decouple, however the 
right-handed translational modulus $\zeta_R$ mixes with
the orientational sector. This is the origin of the name
"heterotic \ntwoo \CPC model".


Now we combine the heterotic \ntwoo deformations of massless
\CPC model studied in \cite{Edalati, SYhet,SYhetlN,BSYhet}
with CP$(N-1)$ model with twisted masses (\ref{sigma22}).
We assume that the model we present in this section 
describes world sheet dynamics of non-Abelian strings
in massive \ntwo QCD with gauge group U$(N)$ and $N$ flavors of quarks
deformed by the superpotential mass term for the adjoint multiplet
\beq
{\mathcal W}_{3+1}=\frac{\mu}{2} \left[{\mathcal A}^2
+  ({\mathcal A}^a)^2\right],
\label{defpo}
\eeq
where $\mu$ is a common mass parameter for the chiral
superfields in \ntwo gauge supermultiplets,
U(1) and SU($N$), respectively. The subscript 3+1 tells us that the deformation 
superpotential (\ref{defpo}) refers to the bulk four-dimensional theory.

Let us note that as was shown in \cite{Edalati,SYhet} 
 the BPS nature of the non-Abelian string string is preserved only
if critical points of the bulk deformation superpotential coincide with quark masses. This is related to the fact that
if the above condition is fulfilled the only  quark scalar fields whose masses are related by \none supersymmetry
to masses of gauge bosons are  excited on the string solution.
Other quark fields with different masses are zero. Once 
above condition is not satisfied these other fields are also
become non-zero. This spoils the BPS-saturation of the string.

Clearly all critical points of the superpotential (\ref{defpo})
are equal to zero. Therefore for a generic non-zero choice of
quark masses the above condition is not satisfied. Thus we expect
that the BPSness of non-Abelian strings is lost. Below in
this paper we will see how this effect is seen from the 
perspective of the effective world sheet theory on the string. We will
see that it manifest itself in the spontaneous breaking of 
the \ntwoo supersymmetry in \CPC model on the string. 
This happen already at the classical level. Note that in the massless case studied in \cite{SYhet,BSYhet} the above 
condition is fulfilled and \ntwoo supersymmetry is preserved
in the world sheet theory on the classical level. Still it turns
out to be spontaneously broken by quantum effects 
\cite{Tongd,SYhet,SYhetlN}.



In the gauged formulation, the two dimensional
\ntwoo \CPC sigma model with twisted
masses  is given by the strong coupling limit $ e ~\to~ \infty $ of
the following U(1) theory
\begin{align}
\label{sigma_full}
% 
\notag
 	\mc{L}_\text{1+1} & ~~=~~
	-\,\frac{1}{8e^2}\,F_{RL}^2  ~+~ \frac{1}{e^2} \left|\p_k \sigma\right|^2 
	~+~ \frac{1}{2e^2}\, D^2
	~+~ \frac{1}{e^2}\, \blar\, i\p_L\, \lar  ~+~  \frac{1}{e^2} \blal\, i\p_R\, \lal
	\\[2mm]
%
\notag
	&
	\hspace{-0.4cm}
	~+~ 2\beta \biggl \lgroup
	\left| \nabla n \right|^2  ~+~ 2 \Bigl| \sigma - 
	\frac{m^l}{\sqrt{2}} \Bigr|^2 \left| n^l \right|^2
	~+~ iD \left( \left|n^l \right|^2 - 1 \right)
	\\
%
\notag
	&
	\hspace{-0.4cm}\phantom{2\beta\lgroup}
	~+~ \bxir\, i\nabla_L \xir  ~+~ \bxil\, i\nabla_R \xil ~+~
	i\sqrt{2}\, \Bigl( \sigma - \frac{m^l}{\sqrt{2}} \Bigr) \ov{\xi}{}_{Rl} \xi_L^l
	~+~ i\sqrt{2}\, \Bigl( \ov{\sigma} - \frac{\ov{m}{}^l}{\sqrt{2}} \Bigr) \ov{\xi}{}_{Ll} \xi_R^l
	\\[2mm]
%
	&
	\hspace{-0.4cm}\phantom{2\beta\lgroup}
	~+~ i\sqrt{2}\, \ov{\xi_{[R}\, \lambda}{}_{L]}\, n
	~-~ i\sqrt{2}\, \nbar\,  \lambda_{[R}\, \xi_{L]}
	\\[2mm]
%
\notag
	&
	\hspace{-0.4cm}\phantom{2\beta\lgroup}
	~+~ \bzr\, i\p_L\, \zr   ~+~   \bff\, \ff
	\\
%
\notag
	&
	\hspace{-0.4cm}\phantom{2\beta\lgroup}
	~-~ 2i\, \frac{\p^2 \hat{\mc{W}}}{\p \sigma^2}\, \blal\, \zr
	~-~ 2i\, \frac{\p^2 \hat{\ov{\mc{W}}}}{\p \ov{\sigma}{}^2}\, \bzr\, \lal
	~+~ 2i\, \frac{\p \hat{\mc{W}}}{\p \sigma}\, \ff
	~+~ 2i\, \bff\, \frac{\p \hat{\ov{\mc{W}}}}{\p \ov{\sigma}}
	\biggr\rgroup ,
	\\[1mm]
%
\notag
	&
	l~=~1,...\,N,
\end{align}
	where $ \hat{\mc{W}}(\Sigma) $ is an arbitrary \ntwot breaking superpotential function.
	In particular, here we consider
	  the latter to be a quadratic deformation
\[
	\hat{\mc{W}}(\Sigma) ~~=~~ \frac{1}{2}\, \delta\, \Sigma^2\,.
\]
The relation of the deformation parameter $\delta$ of the
world sheet theory to the  parameter $\mu$
of the  deformation superpotential (\ref{defpo}) is studied in
\cite{SYhet,BSYhet} \footnote{Strictly speaking, this 
relation is derived in \cite{SYhet,BSYhet} for massless
version of the theory. Since mass deformation is rather independent of \ntwoo deformation we expect that the same 
relation is true for the massive theory. The proof of this is
left for a future work.}
\beq
\delta =
\left\{
\begin{array}{l}\rule{0mm}{5mm}
 {\rm const}\;\,\frac{g_2^2\mu}{m_W}\, , \quad\qquad\;  \mbox{small}\,\,\,
 \mu\,,\\[4mm]
 {\rm const}\,\frac{\mu}{|\mu|}\;
 \sqrt{\ln{\frac{g^2_2\mu}{m_W}}} \, , \qquad\, \mbox{large}\,\,\,\mu\,\, .
 \end{array}
 \right.
\label{deltamu}
\eeq
Here $g_2^2$ is the SU($N$) gauge coupling, 
while $m_W$ is the mass of the SU$(N)$ gauge boson of the bulk theory. The two dimensional deformation parameter is proportional to
the bulk one at small $\mu$, while at large $\mu$ it
acquires a logarithmic dependence.



        We prove supersymmetry of the above action 
        (\ref{sigma_full}) in two steps
	(for now we absorb the factor of $ 2\beta $ into the definition of $ n^l $, $ \xi^l $,
        $ \zeta_R $ and $ \mc{F} $).
        The massive deformation and the heterotic deformation are independent of each other, as we shortly prove.
	We can first consider the theory deformed only by twisted masses, and then add \ntwoo terms.
        Therefore, as the first step we discard the superpotential $ \hat{\mc{W}} $.
        Then the theory splits into two decoupled sectors --- orientational and translational one. 
        The orientational sector (\ref{sigma22}) was 
        already considered in the previous subsection.
   
	


	The translational sector is free
\[
	\bzr\, i\p_L\, \zr ~~+~~ \bff\,\ff
\]
	and invariant under the right-handed supersymmetry:
\begin{align}
%
\notag
        \delta \zr & ~~=~~ ~~~~~ \sqrt{2}\, \eer\, \ff
        &
        \delta \bzr & ~~=~~ ~~~~~ \sqrt{2}\, \beer\, \bff
        \\
%
\label{tsusy}
        \delta \ff & ~~=~~ -\, i\sqrt{2}\, \beer\, \p_L \zr
        &
        \delta \bff & ~~=~~ -\, i\sqrt{2}\, \eer\, \p_L \bzr
        \,.
\end{align}
	Thus the direct sum of the two sectors preserves \ntwoo supersymmetry.
	The final step is to restore the heterotic deformation $ \hat{\mc{W}} $. 
	The \ntwot fields that mix with the translational sector by means
	of $ \hat{\mc{W}} $ are $ \lambda_L $ and $ \sigma $.
	The supertransformations of the latter do not involve the masses $ m^l $, see Eq.~\eqref{msusy},
	therefore the heterotic deformation and the twisted-mass deformation are indeed independent
	from each other.
	Finally, we argue that the $ \hat{\mc{W}} $-terms are invariant under the overall
	right-handed supersymmetry \eqref{msusy} and \eqref{tsusy}, for an arbitrary function $ \hat{\mc{W}}(\sigma) $.
%        In doing so, one notices that when supersymmetry transformations are applied to the Lagrangian
%	the third derivative of $ \hat{\mc{W}} $ does not arise, since $ \delta\sigma ~\propto~ \blal $.

By the analogy with massive non-supersymmetric CP$(N-1)$ model
studied in \cite{GSYphtr} in the large $N$ approximation we expect
presence of two phases in the model (\ref{sigma_full})
separated by a crossover transition.
Namely, the week coupling Higgs phase at large masses and 
the strong
coupling phase at small masses. The detailed study of physics
of the heterotic \CPC model (\ref{sigma_full}) is left for a future work. Here we just comment on the week coupling Higgs phase.

If masses are large $|m^l|\gg \Lambda$ (where $\Lambda$ is
a scale of the worldsheet theory) the coupling constant
$\beta$ is frozen at the scale of order of $|m^l|$.
The theory is in the weak coupling regime and can be studied
perturbatively. We have $N$ vacua in each of them VEV of
$n^l$ is given by
\beq
\langle n^l \rangle =\delta^{ll_0}, \qquad l_0=1,...,N.
\label{higgsn}
\eeq
As follows from (\ref{sigma_full}), in order to find VEV of $\sigma$ in the $l_0$-th vacuum we have to minimize the following potential
\beq
2\left|\sigma -\frac{m^{l_0}}{\sqrt{2}}\right|^2 
+4|\delta|^2\,|\sigma|^2.
\label{sigmapot}
\eeq
This gives
\beq
\langle \sigma \rangle = \frac{m^{l_0}}{\sqrt{2}}\,
\frac{1}{1+2|\delta|^2}.
\label{higgssigma}
\eeq

Substituting this back into the potential (\ref{sigma_full})
we get the vacuum energy and masses of $(N-1)$
elementary fields $n^l$ and $\xi^l$ ($l\neq l_0$). We have
\begin{align}
%
\notag
	E_\text{vac}~~\;\, & ~~=~~ |\gamma|^2\, |m^{l_0}|^2 \,,
	\\
%
\label{hetmass}
	M_\text{ferm}^l\;\,\, & ~~=~~ m^l ~-~ m^{l_0} ~+~ |\gamma|^2\,m^{l_0} \,,
	\\
%
\notag
	| M_\text{bos}^l |^2 & ~~=~~ | M_\text{ferm}^l |^2 ~-~ |\gamma|^4\, |m^{l_0}|^2\,,
	\qquad\qquad
	l\neq l_0,
\end{align}
where we introduced a new parameter $\gamma$ via relation
\beq
\frac{1}{1+2|\delta|^2}\equiv 1-|\gamma|^2.
\label{gammadelta}
\eeq

	Although neither the twisted mass deformation, nor the heterotic deformation by themselves
	do not break supersymmetry completely, when combined, they lead to the spontaneous \ntwoo supersymmetry breaking
	already at the classical level.
 Namely, for non-zero masses 
in each of the Higgs vacua
the vacuum energy is non-zero and the boson masses are different from masses of their fermion superpartners. As we explained
above this is in accord with our expectations which follow
from the bulk theory picture.

	In particular, in the special case where all the masses sit on a circle, 
\[
	m^l ~~=~~ m \cdot e^{2\pi l/N}\,, \qquad\qquad\qquad  l ~=~ 1,...\, N\,,
\]
	the $ N $ vacua become degenerate.

	Note that supersymmetry restores if one of the masses vanishes or $ \gamma $ is set to zero.
	When one of the masses vanishes, the corresponding vacuum becomes supersymmetric,
	Eq.~\eqref{hetmass},
\[
	m^{l_0} ~~=~~ 0  \qquad \Rightarrow \qquad E_\text{vac}^{l_0} ~~=~~ 0\,, \qquad\qquad l_0 ~=~ 1,...\,N\,.
\]
	The theory then becomes a heterotic \CPC model with $ N - 1 $ twisted mass parameters.

	When $ \gamma = 0 $, however, all $ N $ vacua become supersymmetric.
	The heterotic deformation is switched off and one is brought back to a \ntwot CP($N-1$) model
	with a twisted-mass deformation.
	Although formally there are $ N $ twisted mass parameters, it is well-known that the theory 
	has only $ N - 1 $ physical parameters, more precisely the mass differences
\[
	M_\text{bos}^i ~~=~~ M_\text{ferm}^i ~~=~~ m^i ~-~ m^{l_0}\,.
\]
	This can be seen already from the Lagrangian \eqref{sigma_full}, when $ \hat{\mc{W}} ~=~ 0 $. 
	Indeed, in this case there is a freedom of changing $ \sigma $ by an additive constant, 
	$ \sigma ~\to~ \sigma ~-~ M/\sqrt{2} $, simultaneously with shifting the masses, $ m^l ~\to~ m^l ~-~ M $, 
	which effectively decreases the number of mass parameters.

	This is no longer the case when the heterotic deformation is turned back on $ \hat{\mc{W}}(\sigma) ~\neq~ 0 $,
	the shift of $ \sigma $ is no longer a symmetry of the theory.
	There are $ N $ twisted mass parameters all of which are physical. 
	In particular, in each Higgs vacuum at weak coupling, $ N - 1 $ parameters define masses of excitations, while one parameter determines
	the vacuum energy.




        To compare the massive heterotic theory in the gauged sigma model formulation to
        the one in the geometric formulation, we need to exclude all auxiliary fields from the model.
        Therefore it is handy to have $ | n^l |^2 = 1 $, and for that we restore the factor of 
        $ 2\beta $ in the physical sector of the model, see Eq.~\eqref{sigma_full}.
        Also we understand that the factor of $ 2\beta $ would naturally arise if we 
        attempt to derive the sigma model from the vortex string in a four-dimensional bulk theory.

        We now exclude the auxiliary fields from \eqref{sigma_full}. 
        As noted earlier \cite{Edalati,SYhet}, in an \ntwoo theory the right-handed constraint $ \nbar\,\xir = 0 $ is
        violated
\[
	\nbar\, \xir ~~\propto~~ \ov{\delta}\,.
\]
	To restore it we perform a shift of the superorientational variable $ \xir $:
\begin{align*}
%
	& \xi_R' ~~=~~ \xir ~+~ \sqrt{2}\, \ov{\delta}\, n\, \bzr \\
%
	& \ov{\xi}{}_R' ~~=~~ \bxir ~+~ \sqrt{2}\, \delta \nbar\, \zr\,.
\end{align*}
	This obviously changes the normalization of the kinetic term for $ \zr $, which we
	bring back to its canonical form by rescaling $ \zr $
\[
	\zr ~~\to~~ \frac{1}{1 + 2|\delta|^2}\,\zr ~~\equiv~~ ( 1 - |\gamma|^2 )\, \zr\,.
\]
	The result of all this is the following theory
\begin{align}
%
\notag
	\frac{\mc{L}}{2\beta} & ~~=~~ \bzr\, i\p_L\, \zr ~+~ 
		| \p n |^2  ~+~ (\nbar\, \p_k n)^2 ~+~ \bxir\, i\p_L\, \xir ~+~ \bxil\, i\p_R\, \xil
	\\[1mm]
%
\notag
	&~ 
	~-~ (\nbar\, i\p_R n)\, \bxil \xil ~-~  (\nbar\, i\p_L n)\, \bxir \xir
	\\[3mm]
%
\notag
	&~
	~-~ \gamma\, i\p_L\nbar \xir\, \zr ~-~ \ov{\gamma}\, \bxir i\p_L n\, \bzr
	~+~ |\gamma|^2\, \bxil \xil\, \bzr \zr
	\\[3mm]
%
\label{sigma_phys}
	&~
	~+~ (1-|\gamma|^2)\,\bxil\xir\,\bxir\xil ~-~ \bxir\xir\,\bxil\xil
	\\[3mm]
%	
\notag
	&~
	~+~ \sum_l |m^l|^2 \left|n^l \right|^2 
	~-~ i m^l\, \ov{\xi}{}_{Rl} \xi_L^l ~-~ i\ov{m}{}^l\, \ov{\xi}{}_{Ll} \xi_R^l
	\\
%
\notag
	&~
	~-~ i\gamma\, m^l\, \ov{n}{}_l \xi_L^l\, \zr 
	~+~ i\ov{\gamma}\, \ov{m}{}^l\, \ov{\xi}{}_{Ll} n^l\, \bzr
	\\[2mm]
%
\notag
	&~
	~-~ (1-|\gamma|^2)
	\lgr \left|\sum m^l |n^l|^2 \right|^2 
		~-~ i m^l\, |n^l|^2 (\bxir\xil) ~-~ i\ov{m}{}^l\, |n^l|^2(\bxil\xir)
	\rgr ,
	\\[2mm]
%
\notag
	&
	~~~~  l ~=~ 1,...\,N.
\end{align}	
	A few comments are in order for Eq.\eqref{sigma_phys}. 
	Notice that in Eq.\eqref{sigma_full} the massive deformation and the heterotic deformation
	were independent from each other, and we used that fact to prove supersymmetry.
	Now we obtain terms in Eq.\eqref{sigma_phys} which depend both on $\gamma$ and
	$m^l$.
	This happens because we have integrated out the auxiliary fields, which also lead to 
	the consequence that supersymmetry is now realized non-linearly 
	(see \cite{BSYhet} where supersymmetry transformations are written for the
	heterotic CP($N-1$) model).

	With masses set to zero, the model \eqref{sigma_phys} is equivalent to the 
	geometric formulation of the heterotic \ntwoo sigma model \cite{SYhet,BSYhet}.
	We now examine more closely the massive terms in Eq.~\eqref{sigma_phys}.
	The model still contains redundant fields.
	In particular, there are $N$ bosonic fields $n^l$ and $N$ fermionic $\xi^l$,
	whereas the geometric formulation has $N-1$ corresponding variables, see
	Section~\ref{geomf}.
	We can use the constraints
\[
	\ov{n}{}_l\, n^l ~~=~~ 1\,, \qquad \ov{\xi}{}_l\, n^l ~~=~~ 0
\]
	to get rid of some of them, say $ n^N $ and $ \xi^N $.
	We obtain, for the massive terms,
\begin{align}
%
\notag
	\mc{L} & ~~\supset~~ 
	|\gamma|^2\, |m^N|^2  
	\\
%
\notag
	&~
	~+~
	\lgr | m^i - m^N + |\gamma|^2 m^N |^2 ~-~ |\gamma|^4\, |m^N|^2 \rgr\, |n^i|^2 
	\\
%
\notag
	&~
	~-~ (1 - |\gamma|^2)\, \Bigl|\sum_i\, (m^i - m^N)\,|n^i|^2 \Bigr|^2
	\\
%	
\label{sigma_mass}
	&~
	~-~ i\,(m^i - m^N + |\gamma|^2 m^N )\,\ov{\xi}{}_{Ri}\,\xi_L^i
	~-~ i\,(\ov{m}{}^i - \ov{m}{}^N + |\gamma|^2\ov{m}{}^N)\, \ov{\xi}{}_{Li}\, \xi_R^i 
	\\[2mm]
%
\notag
	&~
	~+~ i\,(1-|\gamma|^2)\,(m^i - m^N)\,|n^i|^2(\bxir\,\xil) 
	~+~ i\,(1-|\gamma|^2)\,(\ov{m}{}^i - \ov{m}{}^N)\,|n^i|^2(\bxil\,\xir)
	\\[2mm]
%
\notag
	&~
	~-~ i \gamma\, (m^i - m^N)\, \ov{n}{}_i\, \xi_L^i\, \zr
	~+~ i \ov{\gamma}\, (\ov{m}{}^i - \ov{m}{}^N)\, \ov{\xi}{}_{Li}\, n^i\, \bzr
	\\[2mm]
%
\notag
	&~
	~-~ i |\gamma|^2\, m^N\, \ov{\xi}{}_{RN}\, \xi_L^N
	~-~ i |\gamma|^2\, \ov{m}^N\, \ov{\xi}{}_{LN}\, \xi_R^N\,,
	\qquad\qquad
	i ~=~ 1,...\,N-1\,,
\end{align}
	where we denote
\[
	(\ov{\xi}\, \xi) ~~=~~ \ov{\xi}{}_i\, \xi^i  ~+~  \ov{\xi}{}_N\, \xi^N\,,
	\qquad\qquad\qquad
	i ~=~ 1,...\,N-1\,.
\]

Note, that at large values of $m^l$ all $N$ Higgs vacua (\ref{higgsn}) of the theory  are
still present in the potential (\ref{sigma_mass}).
One of them with $l_0=N$ located at $n^i=0$, $i=1,...,(N-1)$
is easily seen from (\ref{sigma_mass}). Vacuum energy,
masses of fermion and boson elementary excitations match
expression (\ref{hetmass}) for $l_0=0$. Other $(N-1)$
vacua are still present, but not-so-easy to see from (\ref{sigma_mass}).
They are located at $n^{l_0}=1$, $l_0=1,...,(N-1)$.

One can easily see these vacua from different equivalent formulations of the theory which emerge if  we choose to eliminate
field $n^{l_0}$ rather then $n^N$. We stress however, that
all $N$ vacua are, in principle, seen from any of these
equivalent formulations and supersymmetry breaking is spontaneous
rather then explicit.



%%%%%%%%%%%%%%%%%%%%%%%%%%%%%%%%%%%%%%%%%%%%%%%%%%%%%%%%%%%%%%%%%%%%%%%%%%%%%%%%%%
%
%	  		        S E C T I O N
%
%%%%%%%%%%%%%%%%%%%%%%%%%%%%%%%%%%%%%%%%%%%%%%%%%%%%%%%%%%%%%%%%%%%%%%%%%%%%%%%%%%
\section{Geometric Formulation}
\label{geomf}	
	
	The supersymmetric CP($N-1$) model both with heterotic and twisted-mass deformations has
	a geometric description, see \cite{SVZw,SYhet}.
	We start with the pure \ntwot CP($N-1$) model.

	In the gauge formulation of this model one has two sets of $ N - 1 $ (anti)chiral 
	superfields $ \Phi^i $ and $ \ov{\Phi}{}^\bj $, $ i, \bj = 1,..., N-1 $, 
	the lowest components $ \phi^i $, $ \ov{\phi}^\bj $ of which parametrize the target K\"{a}hler
	manifold.
The Lagrangian of the CP($N-1$) model is given by the following sigma model
\[
	\mc{L} ~~=~~ \int\, d^4\theta\, K(\Phi,\ov{\Phi}) ~~=~~ g_{i\bj}\,\p_\mu \phi^i \p_\mu\ov{\phi}{}^\bj
		~+~ \frac{1}{2}\, g_{i\bj}\, \psi^i\, i\overleftrightarrow{\slashed{\nabla}} \ov{\psi}{}^\bj 
		~+~ \frac{1}{4}\, R_{ij\bk\bl}\, \psi^i\psi^j \ov{\psi}{}^\bk \ov{\psi}{}^\bl~,
\]
	where $ K(\phi,\ov{\phi}) $ is the K\"ahler potential, 
	$ g_{i\bj} $ is its K\"ahler metric
\[
	g_{i\bj} ~~=~~ \frac{\p^2 K}{\p\phi^i\,{\p\ov{\phi}{}^\bj}}~,
	\qquad\qquad
	g^{i\bk} ~~=~~ \left(g^{-1}\right)^{\bk i}~,
\]
	$ \nabla_\mu $ the covariant derivative,
\begin{align}
% 
\notag
	(\nabla_\mu \ov{\psi})^\bj & ~~=~~ \left\{ \p_\mu \delta^\bj_{\ \bm} ~+~
						\Gamma_{\bm\bk}^\bj\, \p_\mu(\ov{\phi}{}^\bk) \right\} \ov{\psi}{}^\bm~,
	& \Gamma^\bi_{\bk\bl} & ~~=~~ g^{m\bi}\,\p_\bl\, g_{m\bk}~,
	\\[3mm]
%
\label{covd}
	(\psi \overleftarrow{\nabla}{}_\mu)^i & ~~=~~ 
			\psi^m \left\{ \overleftarrow{\p}{}_\mu\delta^i_{\ m} ~+~
						\Gamma^i_{mk}\, \p_\mu(\phi^k) \right\}~,
	& \Gamma^i_{kl} & ~~=~~ g^{i\bm}\, \p_l\, g_{k\bm}~,
\end{align}
	and $ R_{ij\bk\bl} $ the Riemann tensor 
\[
	R_{ij\bk\bl} ~~=~~ \p_i\,\p_\bk\, g_{j\bl} ~-~ g^{m\bm}\; \p_i\, g_{j\bm}\, \p_\bk g_{m\bl}~.
\]
	For the CP($N-1$) model one chooses the K\"ahler potential
\[
	K(\Phi, \ov{\Phi}) ~~=~~ \ln \lgr 1 ~+~ \ov{\Phi}{}^\bj \delta_{\bj i} \Phi^i \rgr
\]
	which corresponds to the Fubini--Study metric,
\[
	g_{i\bj} ~~=~~ \frac{1}{\chi}\lgr  \delta_{i\bj} ~-~ \frac{1}{\chi}
				  \delta_{i\bi}\,\ov{\phi}^\bi\, \delta_{j\bj}\,\phi^j \rgr,
	\qquad\qquad \text{where~~}
	\chi ~~=~~ 1 ~+~ \ov{\phi}{}^\bj \delta_{\bj i} \phi^i~.
\]
	In this case,
\[
	\Gamma^\bi_{\bk\bl} ~~=~~ -\, \frac{\delta^\bi_{\ (\bk} \delta_{\bl) i}\, \phi^i}{\chi}\,,  
	\qquad\qquad 
	\Gamma^i_{kl} ~~=~~ -\, \frac{\delta^i_{\ (k} \delta_{l)\bi}\,\ov{\phi}{}^\bi}{\chi}\,,
\]
	and the Riemann tensor takes the form
\[
	R_{ij\bk\bl} ~~=~~ -\,g_{i(\bk}\,g_{\bl)j}~.
\]
	
	As was shown in \cite{SYhet},  the \ntwoo deformation of the CP($N-1$) model can be obtained by
	introduction of the right-handed supertranslational modulus $ \zeta_R $ via a ``right-handed'' 
	supermultiplet $ \mc{B} $,
\begin{align*}
%
	\mc{B} & ~~=~~ \lgr \zr ~+~ \sqrt{2}\,\theta_R\mc{F} \rgr \ov{\theta}{}_L~, \\[2mm]
%
	\ov{\mc{B}} & ~~=~~ \theta_L \lgr \bzr ~+~ \sqrt{2}\, \ov{\theta}{}_R \ov{\mc{F}} \rgr.
\end{align*}
	The latter expressions describe superfields covariant only under the right-handed supersymmetry, 
	while explicitly breaking the left-handed one.
	In a sense, $ \mc{B} $ is the analogue of the \ntwoo supermultiplet $\Xi$ in the two-dimensional 
	superfield formalism \cite{Edalati} ---
	the supermultiplet containing only one physical field, which is the supertranslational
	fermionic variable.
	The distinction is that $ \mc{B} $ happens to be a twisted superfield
\[
	D_? \mc{B} ~~=~~ \ov{D}_? \mc{B} ~~=~~ 0\,.
\]
	
	One then constructs the action 
\beq
\label{exte}
	\mc{L}_{(0,2)} ~~=~~ \int\, d^4\theta\, \lgr K(\Phi,\ov{\Phi}) 
		~-~ 2\, \ov{\mc{B}}\,\mc{B}  
		~-~  \sqrt{2}\, \gamma\,\mc{B}\,K  ~-~ \sqrt{2}\, \ov{\gamma}\,\ov{\mc{B}}\,\ov{K} \rgr,
\eeq
	which respects the invariance on the target space CP($N-1$).
	The second term in Eq.~\eqref{exte} generates the kinetic term for $ \zr $, while the last two terms 
	are responsible for the mixing between $ \zr $ and $ \xi_{R,L} $.
	Explicitly, one has,
\begin{align}
%
\notag
	\mc{L}_{(0,2)} & ~~=~~  \bzr\, i\p_L\, \zr 
			~+~ g_{i\bj}\,\p_\mu \phi^i \p_\mu\ov{\phi}{}^\bj
			~+~ \frac{1}{2}\, g_{i\bj}\, \psi^i_R\, i\overleftrightarrow{\nabla}{}_{\!\!L} \ov{\psi}{}^\bj_R 
			~+~ \frac{1}{2}\, g_{i\bj}\, \psi^i_L\, i\overleftrightarrow{\nabla}{}_{\!\!R} \ov{\psi}{}^\bj_L 
	\\[2mm]
%
\label{cpn-1g}
			& 
			~~-~~ \gamma\, g_{i\bj}\, (i \p_L \ov{\phi}{}^\bj)\, \psi_R^i\, \zr
			~-~ \ov{\gamma}\, g_{i\bj}\, \ov{\psi}{}_R^\bj (i \p_L \phi^i)\, \bzr
			~+~ |\gamma|^2\, \bzr\,\zr \cdot ( g_{i\bj}\, \ov{\psi}{}_L^\bj\, \psi_L^i )
	\\[2mm]
%
\notag
			& 
			~~+~~ (1 \!-\! |\gamma|^2)\, (g_{i\bk}\, \ov{\psi}{}_R^\bk\, \psi_L^i)\,
						     (g_{j\bl}\, \ov{\psi}{}_L^\bl\, \psi_R^j)
			~-~ (g_{i\bk}\, \ov{\psi}{}_R^\bk\, \psi_R^i)\, (g_{j\bl}\, \ov{\psi}{}_L^\bl\, \psi_L^j)~.
\end{align}
	
	The geometric form \eqref{cpn-1g} can be related to the gauge formulation (Eq.~\eqref{sigma_phys} with
	vanishing masses), modulo the common factor of $ 2\beta $, via the following stereographic projection
\begin{align}
%
\notag
	n^i & ~~=~~ \frac{\phi^i}{\sqrt{\chi}}~,
	& 
	\ov{n}_\bi & ~~=~~ \frac{\ov{\phi}{}^\bi}{\sqrt{\chi}}~,
\\[2mm]
%
\label{stereo}
	n^N & ~~=~~ \frac{1}{\sqrt{\chi}}~,
	& n^N & ~~\in~~ \mc{R}~,
\\[2mm]
%
\notag
	\xi^i & ~~=~~ \frac{1}{\sqrt{\chi}} \lgr \psi^i ~-~ \frac{(\ov{\phi}\psi)}{\chi}\,\phi^i \rgr,
	& 
	\ov{\xi}{}_\bi & ~~=~~ \frac{1}{\sqrt{\chi}} 
					\lgr \ov{\psi}{}^\bi ~-~ \frac{(\ov{\psi}\phi)}{\chi}\, \ov{\phi}{}^\bi \rgr,
\\[2mm]
%
\notag
	\xi^N & ~~=~~ -\, \frac{(\ov{\phi} \psi)}{\chi^{3/2}}~,
	&
	\ov{\xi}{}_N & ~~=~~ -\, \frac{(\ov{\psi} \phi)}{\chi^{3/2}}~,
\end{align}
	where $	i,\, \bi ~=~ 1, ..., N-1 $ and we shortcut the contractions 
	$ (\ov{\psi} \phi) ~=~ \delta_{i\bj}\, \ov{\psi}{}^\bj \phi^i $.
	Here we chose $ n^N $ to be real given an overall phase freedom of the CP($N-1$) variables $ n^l $.
	Also we have picked up $ n^N $ to be special and equal to $ 1/\sqrt{\chi} $, 
	which corresponds to picking out one of the $ N $ vacua.


\subsection{Twisted Masses}
	The twisted-mass deformation is done by formally lifting the theory to a 4-di\-men\-si\-on\-al space and introducing a set 
	4-dimensional vector superfields $ V^i $:
\begin{equation}
%
\label{Vi}
	V^i ~~=~~ A_1^i\, \theta\sigma_1\ov{\theta} ~+~ A_2^i\, \theta\sigma_2\ov{\theta}\,,
\end{equation}
	with only the ``transverse'' components of the gauge field nonvanishing 
	(we assume that the string is ``oriented'' in the ($x_0$, $x_3$)-plane), and with $ \lambda $ and $ D $ equal to zero.
	These components are constant and define the twisted masses
\begin{equation}
\label{mGi}
	m_G^k ~~=~~ -\, \frac{A_1^k ~+~ i\, A_2^k}{2}\,,
	\qquad\qquad k ~=~ 1, ...\, N-1\,.
\end{equation}
	These vector superfields are precisely the same kind of superfields that could give the twisted masses $m^l$
	to the model \eqref{sigma_full},
	see discussion after Eq.~\eqref{msusy}, only now the number of them is $ N - 1 $ instead of $ N $.
	In particular, as we mentioned, these superfields preserve \ntwot supersymmetry
	after dimensional reduction to two dimensions.
	Until subsection~\ref{sbreaking} we are not going to dwell on the obvious fact that the number of 
	massive parameters in the geometric formulation is less than that of the gauge formulation.
	
	The theory is then gauged with the above vector fields,
\[
	K(\Phi^i, \ov{\Phi}{}^\bj\, V^i) ~~=~~
		\ln \lgr 1 ~+~ \ov{\Phi}{}^\bj\, \delta_{i\bj}\, e^{V^i} \Phi^i \rgr ,
\]	
	with the same action as in Eq.~\eqref{exte},
\beq
\label{Ltw}
	\mc{L}_{(0,2)}^\text{tw.m.} ~~=~~ \int\, d^4\theta\, \lgr K(\Phi,\ov{\Phi}, V^i) 
		~-~ 2\, \ov{\mc{B}}\,\mc{B}  
		~-~  \sqrt{2}\, \gamma\,\mc{B}\,K  ~-~ \sqrt{2}\, \ov{\gamma}\,\ov{\mc{B}}\,\ov{K} \rgr.
\eeq
	The action of the theory \eqref{Ltw} can be calculated by introducing covariantly-chiral superfields
\begin{align*}
%
	X^i & ~~=~~ \Phi^i\,, \\
%
	\ov{X}{}^\bj & ~~=~~ e^{V^\bj} \ov{\Phi}{}^\bj\,,
\end{align*}
	in terms of which the K\"{a}hler potential takes the original form
\[
	K ~~=~~ \ln \lgr 1 ~+~ \ov{X}{}^i X^i \rgr\,.
\]
	It turns out that in the above integral one can freely replace $ D_\alpha $ and $ \ov{D}{}_\alpha $ with
	covariant $ \nabla^{(\bj)}_\alpha $ and $ \ov{\nabla}{}^{(i)}_\alpha $ at any convenient occurrence.
	This makes calculation of \eqref{Ltw} straightforward, and the only obvious difference with the massless
	case comes from the algebra of the covariant derivatives $ \nabla^{(\bj)}_\alpha $ and 
	$ \ov{\nabla}{}^{(i)}_\alpha $, {\it i.e.} from the presence of the constant gauge fields.
	For calculation of the conjugate term $ \sqrt{2}\, \ov{\gamma}\,\ov{\mc{B}}\,\ov{K} $ one can find convenient to 
	use covariantly-antichiral variables
\begin{align*}
%
	Y^i & ~~=~~ e^{V^i} \Phi^i\,, \\
%
	Y^\bj & ~~=~~ \ov{\Phi}{}^\bj\,.
\end{align*}
	Needless to say that the result is obtained from the massless theory by lengthening the space-time derivatives.
	We have,
\begin{align}
%
\notag
	\mc{L}_{(0,2)}^\text{tw.m.} & ~~=~~ 
	g_{i\bj}\,(\nabla_\mu \phi^i)\, (\nabla_\mu\ov{\phi}{}^\bj)
	~+~ \frac{1}{2}\, g_{i\bj}\, \psi^i\, i\overleftrightarrow{\slashed{\nabla}} \ov{\psi}{}^\bj
	\\
%
\notag
	&~
	~-~ \p_i\p_\bk g_{l\bj}\, \psi^i_{R}\, \psi^l_{L}\, \ov{\psi^\bk_{R}\, \psi}{}^\bj_{L}
	~+~ g_{i\bj}\, F^i\,\ov{F}{}^\bj
	~-~ \p_\bk g_{i\bj}\, F^i\, \ov{\psi_R^\bk\, \psi}{}_L^\bj
	~+~ \p_i g_{l\bj}\, \psi_R^i\, \psi_L^l\, \ov{F}{}^\bj
	\\[2mm]
%
\notag
	&~
	~+~ \bzr\, i\p_L\, \zr ~+~ \bff \ff 
	~-~ \gamma\, \ff\, g_{i\bj}\, \psi_R^i\, \bpsi_L^\bj 
	~-~ \bgamma\, \bff\, g_{i\bj}\, \psi_L^i\, \bpsi_R^\bj
	\\[1mm]
%
\label{geomlin}
	&~
	~+~ \gamma\, \ff\, \frac{1}{\chi}\,\phi^\bj\, (\nabla_1^\bj + i \nabla_2^\bj) \bphi^\bj
	~-~ \bgamma\, \bff\, \frac{1}{\chi}\,\bphi^\bi\, (\nabla_1^i - i \nabla_2^i) \phi^i
	\\[1mm]
%
\notag
	&~
	~+~ \gamma\,\zr\cdot g_{i\bj}\, F^i\, \bpsi_L^\bj
	~-~ \bgamma\, \bzr\cdot g_{i\bj}\, \ov{F}{}^\bj\, \psi_L^i
	\\[1mm]
%
\notag
	&~
	~+~ \gamma\, \zr\, \p_i g_{k\bj}\, \psi_R^i\, \psi_L^k \cdot \bpsi_L^\bj
	~+~ \bgamma\, \bzr\, \p_\bk g_{i\bj}\, \ov{\psi_R^\bj\, \psi}{}_L^\bk \cdot \psi_L^i
	\\[2mm]
%	
\notag
	&~
	~+~ \gamma\, \zr\, g_{i\bj}\, \psi_R^i\, i\p_L \bphi^\bj
	~+~ \bgamma\, \bzr\, g_{i\bj}\, \bpsi_R^\bj\, i\p_L \phi^i
	\\[2mm]
%
\notag
	&~
	~+~ \gamma\, \zr\, g_{i\bj}\, \psi_L^i\, (\nabla_1^\bj + i\nabla_2^\bj) \bphi^\bj
	~+~ \bgamma\, \bzr\, g_{i\bj}\, \bpsi_L^\bj\, (\nabla_1^i - i\nabla_2^i) \phi^i
	\,.
\end{align}
	The massive terms here are hidden in the covariant derivatives
\begin{align*}
%
	\nabla_\mu^i & ~~=~~ \p_\mu ~+~ \frac{i}{2}\,A_\mu^i\,,  \\[3mm]
%
	\nabla_\mu^\bj & ~~=~~ \p_\mu ~-~ \frac{i}{2}\,A_\mu^\bj \,,
	&& \!\!\!\!\!\!\! \mu ~=~ 0,...\,3\,,
	\\[3mm]
%
	(\nabla_\mu \ov{\psi})^\bj & ~~=~~ \left\{ \nabla_\mu^\bm \delta^\bj_{\ \bm} ~+~
						\Gamma_{\bm\bk}^\bj\, \nabla_\mu^\bk (\ov{\phi}{}^\bk) \right\} \ov{\psi}{}^\bm~,
	& \Gamma^\bi_{\bk\bl} & ~~=~~ g^{m\bi}\,\p_\bl\, g_{m\bk}~,
	\\[3mm]
%
	(\psi \overleftarrow{\nabla}{}_\mu)^i & ~~=~~ 
			\psi^m \left\{ \overleftarrow{\nabla}{}_\mu^m \delta^i_{\ m} ~+~
						\Gamma^i_{mk}\, \nabla_\mu^k (\phi^k) \right\}~,
	& \Gamma^i_{kl} & ~~=~~ g^{i\bm}\, \p_l\, g_{k\bm}\,.
\end{align*}
	Although the space-time index $\mu$ formally runs through all four values, we understand
	that the derivatives $ \p_\mu $ with respect to the transverse coordinates should be ignored in our two-dimensional
	theory.

	We now need to exclude the auxiliary fields $ F^i $ and $ \ff $, since they have no analogues
	in the gauge formulation of the theory 
	(more precisely, although we did introduce the exactly same $\mc{B}$ superfield into the gauge formulation, 
	the highest component $\ff$ of that multiplet played a different role in Eq.~\eqref{sigma_full} than in Eq.~\eqref{geomlin}).
	Also, we substitute the masses, noticing that the covariant derivatives in Eq.~\eqref{geomlin} 
	enter in convenient combinations, 
\begin{align*}
%
	\nabla_1^\bj ~+~ i\,\nabla_2^\bj & ~~=~~ \phantom{-} i\,m_G^\bj\,,    \\
%
	\nabla_1^i ~-~ i\,\nabla_2^i     & ~~=~~ -i\,\ov{m}{}_G^i\,,
	\qquad\qquad\qquad i,\,\bj ~=~ 1,...\,N-1\,.
\end{align*}

	The $ F^i $-term conditions are the same as in the massless heterotic theory, while $ \ff $-term condition
	gets modified by the masses,
\beq
\label{ffterm}
	\ff ~~=~~ \bgamma\, g_{i\bj}\, \psi_L^i\, \bpsi_R^\bj
	      ~-~ i\, \bgamma\, \ov{m}_G^i \frac{\bphi^i\, \phi^i}{\chi}\,.
\eeq
	We obtain, 
\begin{align}
%
\notag
	\mc{L}_{(0,2)}^\text{tw.m.} & ~~=~~ 
	\bzr\, i\p_L\, \zr \\[3mm]
%
\notag
	&\!\!\!\!\!\!\!\!\!\!\!\!
	~+~ g_{i\bj}\, \p_\mu\phi^i\, \p_\mu\bphi^\bj ~+~ g_{i\bj}\, m_{G\mu}^i\,m_{G\mu}^\bj\, \phi^i\, \bphi^\bj
	\\[2mm]
%
\notag
	&\!\!\!\!\!\!\!\!\!\!\!\!
	~+~ \frac{1}{2}\,g_{i\bj}\,\psi_R^i\,i\overleftrightarrow{\nabla}{}_{\!\!L}^{(0)} \bpsi_R^\bj 
	~+~ \frac{1}{2}\,g_{i\bj}\,\psi_L^i\,i\overleftrightarrow{\nabla}{}_{\!\!R}^{(0)} \bpsi_L^\bj 
	~+~ \frac{i}{2}\,g_{i\bj}\,\psi_L^i\, \overleftrightarrow{m}{}_{\!G} \bpsi_R^\bj
	~+~ \frac{i}{2}\,g_{i\bj}\,\psi_R^i\, \overleftrightarrow{\ov{m}}{}_{\!G} \bpsi_L^\bj
	\\[2mm]
%
\label{sgeom_phys}
	&\!\!\!\!\!\!\!\!\!\!\!\!
	~-~ (g_{i\bj}\, \psi_R^i\, \bpsi_R^\bj)\,(g_{k\bl}\, \psi_L^k\, \bpsi_L^\bl) 
	~+~ (1 \!-\!|\gamma|^2)(g_{i\bj}\, \psi_R^i\, \bpsi_L^\bj)\,(g_{k\bl}\,\psi_L^k\,\bpsi_R^\bl)
	\\[3mm]
%
\notag
	&\!\!\!\!\!\!\!\!\!\!\!\!
	~-~ \gamma\, g_{i\bj}\, i\p_L\bphi^\bj\, \psi_R^i\,\zr 
	~-~ \bgamma\, g_{i\bj}\, \bpsi_R^\bj\, i\p_L\phi^i\, \bzr
	~-~ i\gamma\, g_{i\bj}\, m_G^\bj\,\bphi^\bj\, \psi_L^i\,\zr
	~+~ i\bgamma\, g_{i\bj}\,\ov{m}{}_G^i\, \bpsi_L^\bj\, \phi^i\,\bzr
	\\[3mm]
%
\notag
	&\!\!\!\!\!\!\!\!\!\!\!\!
	~+~ |\gamma|^2\, g_{i\bj}\, \bpsi_L^\bj\, \psi_L^i\, \bzr\, \zr
	\\
%
\notag
	&\!\!\!\!\!\!\!\!\!\!\!\!
	~+~ i\, |\gamma|^2\, (g_{i\bj}\,\psi_L^i\,\bpsi_R^\bj)\, \cdot m_G^k \frac{\bphi^k\, \phi^k}{\chi}
	~+~ i\, |\gamma|^2\, (g_{i\bj}\,\psi_R^i\,\bpsi_L^\bj)\, \cdot \ov{m}{}_G^\bk \frac{\bphi^\bk\,\phi^\bk}{\chi}
	\\
%
\notag
	&\!\!\!\!\!\!\!\!\!\!\!\!
	~+~ |\gamma|^2 \cdot \ov{m}{}_G^\bj\,\frac{\bphi^\bj\,\phi^\bj}{\chi}
			\cdot m_G^i\, \frac{\bphi^i\, \phi^i}{\chi}\,.
\end{align}
	Some comments are due on the notations used in this formula.
	By $ \nabla^{(0)} $ here we have referred to the non-gauge (but still covariant) derivative \eqref{covd}
	of the massless theory.
	Index $ \mu ~=~ 1,\, 2$ of the masses $ m_G^i $ denotes their real and imaginary part correspondingly,
	which is consistent with Eq.~\eqref{mGi}.
	Finally, the matrices $ \overleftrightarrow{m}{}_{\!G} $ act on the spinors in accord with
\begin{align*}
%
	(m_G\, \bpsi_{R,L})^\bj & ~~=~~ 
		\bigl( m_G^\bm\, \delta_{\bm}^{~\bj} ~+~ m_G^\bk\, \Gamma^\bj_{\bm\bk}\, \bphi^\bk \bigr)\, \bpsi_{R,L}^\bm\,,
	\\[2mm]
%
	(\psi_{R,L}\,\overleftarrow{m}{}_{\!G})^i & ~~=~~
		-\, \psi_{R,L}^m\, \bigl( m_G^m\, \delta_m^{~i} ~+~ m_G^k\, \Gamma^i_{mk}\, \phi^k \bigr)\,,
	\\[2mm]
%
	\overleftrightarrow{m}{}_{\!G} & ~~=~~ \overrightarrow{m}{}_{\!G} ~-~ \overleftarrow{m}{}_{\!G}\,,
\end{align*}
	with the identical prescription for the conjugate mass matrix $ \overleftrightarrow{\ov{m}}{}_{\!G} $.

	To this end we can compare the theory \eqref{sgeom_phys} with the gauge formulation of the heterotic sigma
	model with twisted masses \eqref{sigma_phys}.
	One can use the (inverted) correspondence rules \eqref{stereo} for that. 
	For a massless theory, this job was done in \cite{BSYhet}, where it was shown that one formulation exactly
	matches onto the other.
	Therefore, we need to prove the correspondence only for the massive terms.
	We have,
\begin{align}
%
\notag
	\mc{L}_{(0,2)}^\text{tw.m.} & ~~\supset~~ 
	g_{i\bj}\, m_{G\mu}^i\,m_{G\mu}^\bj\, \phi^i\, \bphi^\bj
	~+~ i\,\frac{1}{2}\,g_{i\bj}\,\psi_L^i\, \overleftrightarrow{m}{}_{\!G} \bpsi_R^\bj
	~+~ i\,\frac{1}{2}\,g_{i\bj}\,\psi_R^i\, \overleftrightarrow{\ov{m}}{}_{\!G} \bpsi_L^\bj
	\\[2mm]
%
\notag
	&~
	~-~ i\gamma\, g_{i\bj}\, m_G^\bj\,\bphi^\bj\, \psi_L^i\,\zr
	~+~ i\bgamma\, g_{i\bj}\,\ov{m}{}_G^i\, \bpsi_L^\bj\, \phi^i\,\bzr
	\\[1mm]
%
\notag
	&~
	~+~ i\, |\gamma|^2\, (g_{i\bj}\,\psi_L^i\,\bpsi_R^\bj)\, \cdot m_G^k \frac{\bphi^k\, \phi^k}{\chi}
	~+~ i\, |\gamma|^2\, (g_{i\bj}\,\psi_R^i\,\bpsi_L^\bj)\, \cdot \ov{m}{}_G^\bk \frac{\bphi^\bk\,\phi^\bk}{\chi}
	\\
%
\notag
	&~
	~+~ |\gamma|^2 \cdot \ov{m}{}_G^\bj\,\frac{\bphi^\bj\,\phi^\bj}{\chi}
			\cdot m_G^i\, \frac{\bphi^i\, \phi^i}{\chi}
	~~=~~
	\\
%
\label{sgeom_mass}
	&\!\!\!\!\!\!\!\!
	~=~
	\sum_i |m_G^i|^2 \left|n^i \right|^2 ~-~ (1 \!-\! |\gamma|^2)\, \Bigl| \sum_i m_G^i\, |n^i|^2 \Bigr|^2 
	\\
%
\notag
	&~
	~-~ i\,m_G^i\,\bxi_{Ri}\,\xi_L^i 
	~-~ i\,\ov{m}{}_G^i\, \bxi_{Li}\, \xi_R^i ~+~
	\\[3mm]
%
\notag
	&~
	~+~ i\,(1\!-\!|\gamma|^2)\, m_G^i\,|n^i|^2\, (\bxir\,\xil) 
	~+~ i\,(1\!-\!|\gamma|^2)\, \ov{m}{}_G^i\,|n^i|^2\, (\bxil\,\xir)
	\\[3mm]
%
\notag
	&~
	~-~ i\,\gamma\, m_G^i\,\ov{n}{}_i\, \xi_L^i\, \zr
	~+~ i\,\bgamma\, \ov{m}{}_G^i\, \bxi_{Li}\, n^i\, \bzr\,,
w	\\[3mm]
%
\notag
	&~~~
	\text{$i$, $\bj$, $k$, $\bk$} ~=~ 1,...\, N-1\,.
\end{align}
	Here 
$
	(\ov{\xi}\, \xi) ~~=~~ \ov{\xi}{}_i\, \xi^i  ~+~  \ov{\xi}{}_N\, \xi^N
$,
	similarly to Eq.~\eqref{sigma_mass}.
	Comparing Eq.~\eqref{sgeom_mass} to Eq.~\eqref{sigma_mass}, we can see that the former
	does not match on the latter one exactly.
	It would if we set $ m^N = 0 $ in the gauge formulation.
	As was discussed in section~\ref{gaugef}, this would be a heterotic CP($N-1$) sigma model
	with $ N-1 $ twisted masses, which has a supersymmetric vacuum.
	Eq.~\eqref{sgeom_phys} describes excitations around this vacuum.

	That there was a problem with the number of twisted mass parameters in the geometric formulation 
	was obvious from the beginning, see Eq.~\eqref{mGi}.
	The number of physical fields is the same in both formulations, but the number of masses is not.
	Not only that, the theory \eqref{sgeom_phys} will always be (classically) supersymmetric, whereas
	\eqref{sigma_phys} does break supersymmetry.
	One needs a mechanism to introduce one more mass parameter in the geometric formulation.

%%%%%%%%%%%%%%%%%%%%%%%%%%%%%%%%%%%%%%%%%%%%%%%%%%%%%%%%%%%%%%%%%%%%%
%%%%%%%%%%%%%%%%%%%%%%%%%%%%%%%%%%%%%%%%%%%%%%%%%%%%%%%%%%%%%%%%%%%%%
\subsection{Supersymmetry-breaking Geometric Formulation}
\label{sbreaking}

	Since $ \mc{B} $ is a twisted superfield, one can introduce a twisted superpotential of the form
\[
	\frac{i}{\sqrt{2}}\,a \int \mc{B}\, d^2\wt{\theta} ~~+~~ \text{h.c.},
	\qquad\qquad\qquad\qquad d^2\wt{\theta} ~=~ d\ov{\theta}{}_L\,d\theta_R\,
\]
	(here $ i/\sqrt{2} $ is a convenient normalization).
	This creates a linear in $ \ff $ contribution in the Lagrangian
\[
	\mc{L}_{(0,2)\to(0,0)}^\text{tw.m.} ~~\supset~~ i\,a\,\ff ~+~ i\,\ov{a}\,\bff ~+~ \bff\,\ff ~+ \dots,
\]
	and, correspondingly, changes the $ \ff $-term condition \eqref{ffterm},
\[
	\ff ~~=~~ \bgamma\, g_{i\bj}\, \psi_L^i\, \bpsi_R^\bj 
		~-~ i\, \bgamma\, \ov{m}_G^i \frac{\bphi^i\, \phi^i}{\chi}
		~-~ i\, \ov{a}\,.
\]
	Substituting this into Eq.~\eqref{geomlin} produces (i) vacuum energy, and (ii) mass shifts for
	bosons and fermions.
	Choosing the appropriate value $ a = \gamma\,m^N $, one can match the masses to those of the
	gauge formulation.
	Overall, the supersymmetry-breaking theory has the following mass terms,
\begin{align}
%
\notag
	\mc{L}_{(0,2)\to(0,0)}^\text{tw.m.} & 
	~~=~~ \int\, d^4\theta\, \lgr K(\Phi,\ov{\Phi}, V^i) 
		~-~ 2\, \ov{\mc{B}}\,\mc{B}  
		~-~  \sqrt{2}\, \gamma\,\mc{B}\,K  ~-~ \sqrt{2}\, \ov{\gamma}\,\ov{\mc{B}}\,\ov{K} \rgr
	\\
%
\notag
	&\!\!\!\!\!\!\!\!
	~+~ \frac{i}{\sqrt{2}} \int d^2\wt{\theta} \cdot \gamma\,m^N\,\mc{B} 
	~+~ \frac{i}{\sqrt{2}} \int d^2\ov{\wt{\theta}} \cdot \bgamma\, \ov{m}{}^N\, \ov{\mc{B}}
	~~\supset~~
	\\
%
\notag
	&\!\!\!\!\!\!\!\!
	~\supset~~
	|\gamma|^2\, \left|m^N\right|^2 
	~+~ g_{i\bj}\, m_{G\mu}^i\, m_{G\mu}^\bj\, \phi^i\, \bphi^\bj
	~+~ |\gamma|^2\, \lgr \ov{m}{}^N m_G^i ~+~ m^N \ov{m}{}_G^i \rgr \frac {\phi^i\,\bphi^i}{\chi}
	\\[1mm]
%
\label{geom_mass}
	&\!\!\!\!\!\!\!\!
	~+~ |\gamma|^2 \cdot \ov{m}{}_G^\bj\,\frac{\bphi^\bj\,\phi^\bj}{\chi}
			\cdot m_G^i\, \frac{\bphi^i\, \phi^i}{\chi}
	\\[1mm]
%
\notag
	&\!\!\!\!\!\!\!\!
	~+~ i\,\frac{1}{2}\,g_{i\bj}\,\psi_L^i\, \overleftrightarrow{m}{}_{\!G} \bpsi_R^\bj
	~+~ i\,\frac{1}{2}\,g_{i\bj}\,\psi_R^i\, \overleftrightarrow{\ov{m}}{}_{\!G} \bpsi_L^\bj
	\\[3mm]
%
\notag
	&\!\!\!\!\!\!\!\!
	~+~ i\,|\gamma|^2\,m^N\,g_{i\bj}\,\psi_L^i\, \bpsi_R^\bj
	~+~ i\,|\gamma|^2\,\ov{m}{}^N\,g_{i\bj}\,\psi_R^i\, \bpsi_L^\bj
	\\[4mm]
%
\notag
	&\!\!\!\!\!\!\!\!
	~-~ i\gamma\, g_{i\bj}\, m_G^\bj\,\bphi^\bj\, \psi_L^i\,\zr
	~+~ i\bgamma\, g_{i\bj}\,\ov{m}{}_G^i\, \bpsi_L^\bj\, \phi^i\,\bzr
	\\[1mm]
%
\notag
	&\!\!\!\!\!\!\!\!
	~+~ i\, |\gamma|^2\, (g_{i\bj}\,\psi_L^i\,\bpsi_R^\bj)\, \cdot m_G^k \frac{\bphi^k\, \phi^k}{\chi}
	~+~ i\, |\gamma|^2\, (g_{i\bj}\,\psi_R^i\,\bpsi_L^\bj)\, \cdot \ov{m}{}_G^\bk \frac{\bphi^\bk\,\phi^\bk}{\chi}
	\,.
\end{align}
	
	Under the stereographic projection \eqref{stereo} this turns into
\begin{align*}
%
	\mc{L}_{(0,2)\to(0,0)}^\text{tw.m.} & 
	~~\supset~~ 
	|\gamma|^2\, \left|m^N\right|^2 
	\\
%
	&\!\!\!\!\!\!\!\!
	~+~
	\lgr |m_G^i|^2 ~+~
		|\gamma|^2 \Bigl\{ \ov{m}{}^N m_G^i ~+~ m^N \ov{m}{}_G^i \Bigr\} \rgr \left|n^i \right|^2 
	~-~ (1 \!-\! |\gamma|^2)\, \Bigl| \sum_i m_G^i\, |n^i|^2 \Bigr|^2 
	\\
%
	&\!\!\!\!\!\!\!\!
	~-~ i\, ( m_G^i \,+\, |\gamma|^2 m^N )\, \bxi_{Ri}\, \xi_L^i
	~-~ i\, (\ov{m}{}_G^i \,+\, |\gamma|^2 \ov{m}{}^N )\, \bxi_{Li}\, \xi_R^i
	\\[3mm]
%
	&\!\!\!\!\!\!\!\!
	~+~ i\,(1\!-\!|\gamma|^2)\, m_G^i\,|n^i|^2\, (\bxir\,\xil) 
	~+~ i\,(1\!-\!|\gamma|^2)\, \ov{m}{}_G^i\,|n^i|^2\, (\bxil\,\xir)
	\\[3mm]
%
	&\!\!\!\!\!\!\!\!
	~-~ i\,\gamma\, m_G^i\,\ov{n}{}_i\, \xi_L^i\, \zr
	~+~ i\,\bgamma\, \ov{m}{}_G^i\, \bxi_{Li}\, n^i\, \bzr\,,
	\\[3mm]
%
	&\!\!\!\!\!\!\!\!
	~-~ i |\gamma|^2\, m^N\, \ov{\xi}{}_{RN}\, \xi_L^N
	~-~ i |\gamma|^2\, \ov{m}^N\, \ov{\xi}{}_{LN}\, \xi_R^N\,,
	\qquad\qquad
	i ~=~ 1,...\,N-1\,.
\end{align*}
	We observe that this Lagrangian matches exactly
	onto the gauge formulation of the heterotic massive sigma model \eqref{sigma_mass}, provided that
	we accept
\[
	m_G^i ~~=~~ m^i ~-~ m^N \,.
\]

As an additional check we now show that in the 
large  mass limit $|m^l|\gg \Lambda$ we can
recover all $N$ Higgs vacua (\ref{higgsn}) obtained in
gauge formulation from the geometric formulation (\ref{geom_mass}). One of these vacua with $l_0=N$ corresponds
to $\phi^i=0$. Other $(N-1)$ vacua located at $\phi^{l_0}\to\infty$, $l_0=1,...,(N-1)$ as it is seen from 
(\ref{stereo}). Vacuum energies, boson and fermion masses 
in these vacua exactly match expressions (\ref{hetmass})
obtained in the gauge formulation.



%	As we noted, the theory \eqref{geom_mass} describes %excitations around the $N$-th vacuum.

	While it took us some effort to prove \ntwoo supersymmetry of the theory \eqref{sigma_full},
	the geometric formulation of this theory 
\begin{align*}
	\mc{L}_{(0,2)\to(0,0)}^\text{tw.m.} & 
	~~=~~ \int\, d^4\theta\, \lgr K(\Phi,\ov{\Phi}, V^i) 
		~-~ 2\, \ov{\mc{B}}\,\mc{B}  
		~-~  \sqrt{2}\, \gamma\,\mc{B}\,K  ~-~ \sqrt{2}\, \ov{\gamma}\,\ov{\mc{B}}\,\ov{K} \rgr
	\\
%
	&~
	~+~ \frac{i}{\sqrt{2}} \int d^2\wt{\theta} \cdot \gamma\,m^N\,\mc{B} 
	~+~ \frac{i}{\sqrt{2}} \int d^2\ov{\wt{\theta}} \cdot \bgamma\, \ov{m}{}^N\, \ov{\mc{B}}
\end{align*}
	is manifestly supersymmetric.
	
	
\section{Conclusions}
	
	
\section*{Acknowledgments}
This work  is supported in part by DOE grant DE-FG02-94ER408. 
The work of A.Y. was  supported 
by  FTPI, University of Minnesota, 
by RFBR Grant No. 09-02-00457a 
and by Russian State Grant for 
Scientific Schools RSGSS-11242003.2.


%%%%%%%%%%%%%%%%%%%%%%%%%%%%%%%%%%%%%%%%%%%%%%%%%%%%%%%%%%%%%%%%%%%%%%%%%%%%%%%%%%
%%%%%%%%%%%%%%%%%%%%%%%%%%%%%%%%%%%%%%%%%%%%%%%%%%%%%%%%%%%%%%%%%%%%%%%%%%%%%%%%%%
%
%                            B I B L I O G R A P H Y
%
%%%%%%%%%%%%%%%%%%%%%%%%%%%%%%%%%%%%%%%%%%%%%%%%%%%%%%%%%%%%%%%%%%%%%%%%%%%%%%%%%%
%%%%%%%%%%%%%%%%%%%%%%%%%%%%%%%%%%%%%%%%%%%%%%%%%%%%%%%%%%%%%%%%%%%%%%%%%%%%%%%%%%
\small
\begin{thebibliography}{99}
\itemsep -2pt



\bibitem{Edalati}
  M.~Edalati and D.~Tong,
  %``Heterotic vortex strings,''
  JHEP {\bf 0705}, 005 (2007)
  [arXiv:hep-th/0703045].
  %%CITATION = JHEPA,0705,005;%%

\bibitem{SYhet}
  M.~Shifman and A.~Yung,
  %``Heterotic Flux Tubes in N=2 SQCD with N=1 Preserving Deformations,''
  Phys.\ Rev.\  D {\bf 77}, 125016 (2008)
  [arXiv:0803.0158 [hep-th]].
  %%CITATION = PHRVA,D77,125016;%%

\bibitem{BSYhet}
  P.~A.~Bolokhov, M.~Shifman and A.~Yung,
  %``Description of the Heterotic String Solutions in U(N) SQCD,''
  [arXiv:0901.4603 [hep-th]].
  %%CITATION = ARXIV:0901.4603;%%
  
\bibitem{SVZw}
  M.~Shifman, A.~Vainshtein and R.~Zwicky,
  %``Central charge anomalies in 2D sigma models with twisted mass,''
  J.\ Phys.\ A  {\bf 39}, 13005 (2006)
  [arXiv:hep-th/0602004].
  %%CITATION = JPAGB,A39,13005;%%
  
     
   

\bibitem{HT1}
A.~Hanany and D.~Tong,
%``Vortices, instantons and branes,''
JHEP {\bf 0307}, 037 (2003)
[hep-th/0306150].
%%CITATION = HEP-TH 0306150;%%

\bibitem{ABEKY}
R.~Auzzi, S.~Bolognesi, J.~Evslin, K.~Konishi and A.~Yung,
%{\em Non-Abelian superconductors: Vortices and confinement in N = 2
%SQCD,}
Nucl.\ Phys.\ B {\bf 673}, 187 (2003)
[hep-th/0307287].
%%CITATION = HEP-TH 0307287;%%

 \bibitem{SYmon}
M.~Shifman and A.~Yung,
%``Non-Abelian string junctions as confined monopoles,''
Phys.\ Rev.\ D {\bf 70}, 045004 (2004)
[hep-th/0403149].
%%CITATION = HEP-TH 0403149;%%

\bibitem{HT2}
A.~Hanany and D.~Tong,
%``Vortex strings and four-dimensional gauge dynamics,''
JHEP {\bf 0404}, 066 (2004)
[hep-th/0403158].
%%CITATION = HEP-TH 0403158;%%

\bibitem{Trev}
D.~Tong,
{\em TASI Lectures on Solitons,}
  arXiv:hep-th/0509216.
  %%CITATION = HEP-TH/0509216;%%
  
\bibitem{SYrev}
M.~Shifman and A.~Yung,
%{\sl Supersymmetric Solitons,}
Rev.\ Mod.\ Phys. {\bf 79} 1139 (2007)
[arXiv:hep-th/0703267].
  %%CITATION = HEP-TH/0703267;%%
M.~Shifman and A.~Yung,
{\sl Supersymmetric Solitons,}
Cambridge University Press, 2009
 
\bibitem{Jrev}
  M.~Eto, Y.~Isozumi, M.~Nitta, K.~Ohashi and N.~Sakai,
  %``Solitons in the Higgs phase: The moduli matrix approach,''
  J.\ Phys.\ A  {\bf 39}, R315 (2006)
  [arXiv:hep-th/0602170].
  %%CITATION = JPAGB,A39,R315;%%
   
\bibitem{Trev2}
D.~Tong,
{\em Quantum Vortex Strings: A Review,}
  arXiv:0809.5060 [hep-th].
  %%CITATION = ARXIV:0809.5060;%%
  



 \bibitem{SYhetlN}
M.~Shifman and A.~Yung,
%''Large-N Solution of the Heterotic N=(0,2) Two-Dimensional CP(N-1) Model''
Phys.\ Rev.\ D {\bf 77}, 125017 (2008)
[arXiv:hep-th/0803.0698].
%%CITATION = HEP-TH 0803.0698;%%

\bibitem{W79}
E.~Witten,
%``Instantons, The Quark Model, And The 1/N Expansion,''
Nucl.\ Phys.\ B {\bf 149}, 285 (1979).
%%CITATION = NUPHA,B149,285;%%


\bibitem{Tohetdyn}
  D.~Tong,
  %``The quantum dynamics of heterotic vortex strings,''
  JHEP {\bf 0709}, 022 (2007)
  [arXiv:hep-th/0703235].
  %%CITATION = JHEPA,0709,022;%%


  
\bibitem{HaHo}
A.~Hanany and K.~Hori,
  %``Branes and N = 2 theories in two dimensions,''
  Nucl.\ Phys.\  B {\bf 513}, 119 (1998)
  [arXiv:hep-th/9707192].
  %%CITATION = NUPHA,B513,119;%%
  
\bibitem{Dorey}
N.~Dorey,
%``The BPS spectra of two-dimensional
%supersymmetric gauge theories
%with  twisted mass terms,''
JHEP {\bf 9811}, 005 (1998) [hep-th/9806056].
%%CITATION = HEP-TH 9806056;%%

\bibitem{Tongd}
  D.~Tong,
  %``The quantum dynamics of heterotic vortex strings,''
  JHEP {\bf 0709}, 022 (2007)
  [arXiv:hep-th/0703235].
  %%CITATION = JHEPA,0709,022;%%
  
\bibitem{GSYphtr}
A.~Gorsky, M.~Shifman and A.~Yung,
%``Higgs and Coulomb/confining phase in ``twisted-mass'' deformed $CP(N-1)$ model''
 Phys.\ Rev.\  D {\bf 73}, 065011 (2006)
  [hep-th/0512153].
  %%CITATION = PHRVA,D73,065011;%%


\end{thebibliography}


\end{document}


	We have excluded the field $ n^N $ completely, while we still left $ \xi^N $ in some places,
	few enough so we can recover the masses of the physical fields. 
	The first three lines of Eq.~\eqref{sigma_mass} display the scalar potential of the theory,
	the fourth line shows the fermionic mass terms, while the rest of the terms are 
	fermionic interactions.

	From Eq.~\eqref{sigma_mass} one easily reads off the masses of the physical fields
	and the vacuum energy:
\begin{align}
%
\notag
	E_\text{vac}~~\;\, & ~~=~~ |\gamma|^2\, |m^N|^2 \,,
	\\
%
\label{hetmass}
	M_\text{ferm}^i\;\,\, & ~~=~~ m^i ~-~ m^N ~+~ |\gamma|^2\,m^N \,,
	\\
%
\notag
	| M_\text{bos}^i |^2 & ~~=~~ | M_\text{ferm}^i |^2 ~-~ |\gamma|^4\, |m^N|^2\,,
	\qquad\qquad
	i ~=~ 1,...\, N-1\,.
\end{align}
	Although neither the twisted mass deformation, nor the heterotic deformation by themselves
	do not break supersymmetry completely, when combined, they lead to supersymmetry breaking
	already at the classical level.
	We have chosen to eliminate variable $n^N$, but in general it could have been any of the $n^l$.
	Hence the vacuum energy equals to $ |\gamma|^2\, |m^l|^2 $, with $ l = 1,...\, N $ labeling the 
	corresponding vacuum.

	All these vacua are necessarily Higgs vacua, and one has to take the masses large enough
	$ m^l \gg \Lambda $ in order to study them.
	In principle, there is also a strong-coupling phase in this theory, but it cannot
	be seen in the approach of Eq.~\eqref{sigma_mass}, nor can it be seen in the geometric formulation
	of the theory, which we are passing on to.	
	The problem of the strong-coupling phase is left for future work.

